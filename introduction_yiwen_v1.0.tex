\section{Introduction}
\label{sec.introduction}

Privileged code is an essential component of modern computer systems but 
presents a number of security challenges. The privileged code itself is vulnerable 
to attacks and other parts of a system could be damaged when vulnerabilities in 
that privileged code are exploited.

Failures in the TCB allow more impactful crashes (complete system failures), privilege escalation, etc.
(Mention zero-days, impact, etc.)
Decreasing the feasibility of exploitation of the kernel bugs, especially privilege escalation, 
would be a substantial step toward stronger security for computer systems.

Bugs in operating system kernels have motivated development of a diverse set of
technologies to attempt to reduce these risks, such as OS virtualization, system
call filtering, library OSes, etc. Unfortunately, these technologies also 
harbor vulnerabilities that are also exploitable. Even with these technologies in place,
applications from the user space could still have access to a portion of the kernel that might contain 
bugs and is risky to be exposed. 

One contributing factor to security problems associated with these existing technologies and potential
new designs is that it is still unknown which portions of the privileged code are
safe to expose, and which portions would be be vulnerable. 
One key missing puzzle is a standard for quantifying the safety (or risky) levels of the privileged code. 
For example, is it good practice to minimize \textit{the number of lines of code}? cite[?] 
Is \textit{the number of API calls} a good metric for security? cite[Bascule]   
To our best knowledge, there is no quantitative metric to evaluate the privileged code and provide 
insights into how to design a secure system that only interact with the privileged code in a safe way.

In this paper, we provide a quantitative measure that shows
certain areas of privileged code are likely to have flaws.  
The measure examines lines of privileged code that are executed by popular 
programs. The intuition behind this metric is that kernel code 
that is rarely executed is less likely to be rigorously tested and is thus more likely to contain bugs. 
We examine 40 historical  kernel bugs from Linux over the past 5 years and find out that 
``commonly'' executed code metric has a strong correlation with being bug-free. 

By optimizing our metric, we devise the \emph{safely-reimplement}
architecture, which minimizes the amount of risky privileged code that is
executed.  
Risky functionality is itself implemented in a sandbox with
a small trusted-computing base (TCB). 
This additional level of sandboxing provides an outlet for risky functionality, without which
legacy programs will not run, while containing security flaws in this code. 

The contributions of this paper are as follows:
\begin{itemize}
\item We proposed a novel metric for evaluating security level of the privileged code. 
Our metric consists of ``commonly'' executed code in the kernel at the level of ``lines of code''. 

\item We quantitatively measured the privileged code using our metric, and labeled certain portions of 
the kernel as ``safe" and certain portions of the kernel as ``risky''.

\item We designed a novel secure architecture that comes from examining and leveraging our metric. 
Using this new architecture, we implemented a sandbox we called Lind, which provides more secure environment
for running legacy applications. 

\item Evaluation shows that implementation of our sandbox Lind only has the potential to trigger 2.5\% of zero-day 
vulnerabilities we examined, while systems built without using our security metric have more chances to trigger 
vulnerabilities. (illustrate with data from running Graphene and Virtualbox)
\end{itemize}