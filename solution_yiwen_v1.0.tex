\section{Solution: A Quantitative Security Metric}
\label{sec.solution}

We propose a security metric to assess and evaluate the kernel in a quantitative way. 
Our metric focuses on examining the precise lines of code in the kernel that are executed 
when running applications in the user space. We define the \textit{kernel trace} to be the 
set of lines of code in the kernel that are executed by running applications. With the \textit{kernel trace}, 
we are able to determine the \textit{kernel coverage} of running applications. The \textit{kernel coverage} 
refers to the percentage proportion of the \textit{kernel trace} to the entire kernel. 

To determine whether the generated \textit{kernel trace} is safe or risky, we examined 40 historical bugs from
the Linux kernel. We identified the lines of code in the kernel that have the potential to trigger each of the bugs.
We define the lines of code in the kernel that have the potential to trigger bugs as \textit{trap trace}.
We then compare the \textit{kernel trace} against the \textit{trap trace}. If there is correlation between 
them, we label the \textit{kernel trace} as \textit{risky trace}. If there is no correlation found between them, 
we label the \textit{kernel trace} as \textit{safe trace}.

\subsection{The Goal of Our Metric}
We propose our metric to achieve the following goals:
\begin{itemize}
\item Define a standard way to conduct quantitative measurement of the lines of code being executed 
in the kernel. 
\item Use our metric to acquire precise information about the \textit{kernel trace} and \textit{kernel coverage} 
of running different applications, to gain better understanding of which portion of the kernel is reachable and 
reachable by running popular applications. 
\item Determine the \textit{risky trace} and \textit{safe trace} from the \textit{kernel trace} generated by
running a set of different applications. Therefore, construct a picture of the the safe portion and risky portion
of the kernel. 
\end{itemize}

\subsection{Intuition Behind Our Metric}
There are some fundamental questions that we need to answer regarding our metric:
\begin{itemize}
\item \textbf{\textit{Why would you need this new metric?}}
\item \textbf{\textit{How did you come up with this metric?}}
\item \textbf{\textit{Can you prove this metric is useful?}}
\end{itemize}

To answer those questions:
\begin{itemize}
\item \textbf{\textit{Our metric is necessary:}} previously, there are very limited ways to measure the kernel
code, in terms of which portion is executed, let along any quantitative ways. Our metric fills the hole in this 
area, and provide a precise way to conduct quantitative measurement.
\item \textbf{\textit{Our metric is reasonable:}} examining the exact lines of code is fine-grained, and therefore
more precise. It provides insights into what is really going on inside the kernel. 
\item \textbf{\textit{Our metric is useful:}} the data of the \textit{kernel trace} and \textit{kernel coverage}
can provide precise statistics of the reachable kernel, which itself is valuable for gaining better understanding
of the kernel. In addition, the information gained by using our metric can help with new designs that expose 
the kernel in a more secure way. 
\end{itemize}

\subsection{Quantitative Measurement of the Kernel Using Our Metric}
Now, we use the metric that we proposed to conduct quantitative measurement of the kernel. 
To be more precise, we would like to gain the \textit{kernel trace} and \textit{kernel coverage} of running 
different set of applications. We can then determine the reachable kernel and reachable kernel by 
running popular applications. Finally, we compare the \textit{kernel trace} we generated against the 
\textit{trap trace} composed of bugs to label \textit{risky trace} and \textit{safe trace}. 

\subsubsection{\textit{Kernel Trace} and \textit{Kernel Coverage} of basic system calls}
Results with running a set of basic system calls in C program. Results with running the system call fuzzer. 

\subsubsection{\textit{Kernel Trace} and \textit{Kernel Coverage} of using popular applications}

\subsubsection{\textit{Kernel Trace} and \textit{Kernel Coverage} of daily user behavior}

\subsubsection{\textit{Security Analysis of the Kernel}}
Compare the \textit{kernel trace} against the \textit{trap trace}.
Label the \textit{risky trace} and the \textit{safe trace}.