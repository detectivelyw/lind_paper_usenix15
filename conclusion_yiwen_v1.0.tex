\section{Conclusion}
\label{sec.conclusion}


Isolating untrusted user applications from the underlying kernel is desirable, in order to protect the privileged code and 
avoid the exploitation of bugs. We proposed a new ``safely-reimplement'' model to achieve stronger isolation. 
Base upon our new strategy, we designed and implemented our system, Lind, which securely reconstruct complex 
yet essential OS functionality inside a dual-layer sandbox. The sandbox itself is designed to 
have a minimized trusted computing base (TCB). 

Our evaluation has shown that complex legacy applications ran in Lind with reasonable overhead, 
thus making Lind a practical tool that may be widely deployed. 
Furthermore, running applications in Lind only touched a small portion of commonly used kernel paths, 
which are not likely to trigger kernel bugs. Evaluation results have proved that Lind provides 
better isolation and stronger protection to OS kernels and the entire system.  

Lind is a viable proof-of-concept platform for securely running legacy applications, while protecting the privileged code.