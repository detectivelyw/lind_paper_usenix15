\begin{abstract}


\par
Operating System (OS) kernels are vulnerable and exploitable. Without proper protection, numerous bugs and malicious code in applications can easily exploit vulnerabilities in the underlying OS kernel and endanger the entire system. Unfortunately, previous efforts still allow untrusted code to have access to substantial kernel footprint, which will lead to the exploitation of kernel bugs. In our work, we proposed a new ``safely re-implement'' strategy. Using this new isolation model, we designed and implemented our system Lind, which securely reconstructed complex yet essential OS functionality inside a dual-layer sandbox. The sandbox itself has minimized trusted computing base. Running legacy applications in Lind has reasonable overhead, and will only touch a small portion of commonly used kernel path that is not likely to trigger hidden bugs. Lind provides better isolation and stronger protection to the underlying kernel, thus making it a highly desirable tool for securely running legacy applications.  


\end{abstract}
