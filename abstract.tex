\begin{abstract}

An Operating System (OS) kernel is a trusted part of modern computer systems, 
yet despite substantial effort, kernels still contain bugs and are vulnerable to many types of 
attacks. Many previous approaches, including OS virtualization, system call filtering, and library 
OSes, have been deployed to provide better security. 
However, bugs continue to exist and remain exploitable even with these approaches in place. 

In this paper, we quantitatively measure and evaluate different lines of code in the kernel 
based upon how often these lines are executed within popular applications. 
Analyzing a history of Linux kernel flaws, we propose our �commonly� executed code metric 
and show that it has a strong correlation with being bug-free. 
Our metric provides insights into how to design a security system to provide stronger security 
to computer systems. 
For example, we used our metric to come up with a design that minimizes the trust placed 
in risky privileged code. This is done by containing, within a sandbox, risky code that is needed 
to support legacy programs. The sandbox is built so that it has a small TCB that minimizes 
the use of risky code in the kernel. Using this technique, we implemented a sandbox we call Lind. 
By running popular Linux packages and legacy programs inside Lind, we evaluated the 
kernel traces that were generated and compared them against kernel traces produced 
when running programs in environments other than Lind, such as Graphene and VirtualBox. 
Our evaluation demonstrates that designs, such as our Lind prototype, provide security benefits 
by minimizing the use of code that is not commonly executed.

\end{abstract}