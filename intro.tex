\section{Introduction}
\label{sec.introduction}

Despite substantial effort, privileged code in modern systems is vulnerable 
to many types of attacks. \linda{I don't think the first sentence says all that 
much, or else, what it says is rather vague. Here are some suggestions: 
Privileged code is an essential component of modern computer systems but 
presents a number of security challenges. The privileged code itself is vulnerable 
to attacks and other parts of a system could be damaged when vulnerabilities in 
that privileged code are exploited. Our work focuses how to better protect 
privileged code, which in turn will help protect other parts of the system.}

 \cappos{Emphasize failures in the TCB allowing more 
impactful crashes (complete system failures), privilege escalation, etc.
Mention zero-days, impact, etc.}
Decreasing the feasibility of these attacks, especially privilege escalation, 
would be a substantial step toward security as a whole. \linda{"security as a whole" 
seems rather broad to me}

Bugs in operating system kernels have motivated development of a diverse set of
technologies to attempt to reduce these risks.  \cappos{OS virtualization, system
call filtering, library OS, etc. }  Unfortunately, these technologies also 
harbor vulnerabilities that are also exploitable.
\yanyan{add more details to explain why.}

One contributing factor to security problems associated with these new designs is
that it is still unknown what portions of privileged code are
safe to expose, and which portions would \sout{present risk} \linda{be be vulnerable}. 
Missing is a standard for quantifying such safety (or risky) levels. For example, 
\sout{should one}\linda{is it good practice to} minimize 
\textit{the number of lines of code}? \cappos{cites}  Is \textit{the number of API 
calls} a good metric for security? \cappos{cite Bascule}   
\cappos{Anything else people have mentioned?}

In this paper, we provide a quantitative measure that shows
areas of privileged code are likely to have flaws.  \linda{it isn't clear to me here what you 
mean by "areas of privileged code"  -- I think the wording just needs to be more precise, but
I'm not sure how, exactly}  \cappos{There was a
paper that did this using number of developers editing a module.  There
are likely others in this area that need to be part of related work.}
The measure examines lines of privileged code that are executed by popular 
programs. \yanyan{=========== attention.. please DONOT use 'touch'. It sounds 
very informal. ===========} The intuition behind this metric is that kernel code 
that is rarely exercised \linda{does one exercise code?} is less likely to be \sout{well} 
rigorously tested and is thus more likely to contain bugs.  We examine historical privilege 
escalation bugs from Linux over the past XX years and find that...

By optimizing this metric  \cappos{need a name} we devise the \emph{safely-reimplement}
architecture, which minimizes the amount of risky privileged code that is
executed.  
Risky functionality is itself implemented in a sandbox with
a small trusted-computing base.  
\yanyan{instead of 'risky functionatilty', should it be 'privileged operations' instead? If you 
want to quantify and minimize the risk of privilege escalation attacks (and convince me 
and others we indeed only handle privilege escalation attacks), then this makes sense.}
This additional level of sandboxing provides an outlet for risky functionality, without which
legacy programs will not run, while containing security flaws in this code. \linda{I think the logic
of this sentence is off} 
\cappos{ I need to figure out how / whether to tie this in: 
We also analyze which functionality must be retained in privileged code.}

The contributions of this paper are as follows:

Quantitative metric for security in this context\yanyan{what context?}

Novel architecture that comes from examining this metric.  Examination of
what functionality must exist where.

Implementation / Eval of sandbox using this philosophy. \yanyan{what philosophy?} 
Only X\% of zero days are exploitable w/ this vs YY\% or ZZ\%.



