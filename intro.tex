\section{Introduction}
\label{sec.introduction}

Despite substantial effort, modern systems are vulnerable to privilege 
escalation attacks.  \cappos{Mention zero-days, impact, etc.}
Decreasing the feasibility of priviledge escalation would be a substantial
step toward security as a whole.
\yanyan{are we only targeting at privilege escalation and why?}
\cappos{What sorts of bugs do we protect from?  We certainly don't protect 
against bugs in apps.  We really want to stop any sort of bug that lets a 
malicious program exploit flaws in the underlying kernel.  Perhaps privilege 
escalation is a simplified subset, but it does capture what we are doing.}

Bugs in operating system kernels has led to the development of a diverse set of
technologies to attempt to reduce the risk.  \cappos{OS virtualization, system
call filtering, library OS, etc. }  Unfortunately, these technologies also 
have vulnerabilities that are also exploitable.
\yanyan{add more details to explain why.}

One contributing factor to the security problems with these new designs is
that it is still unknown what portions of privileged code are
safe to expose, and which portions would present risk. There lacks a standard 
to quantify such safety (or risky) levels. For example, should one minimize 
\textit{the number of lines of code}? \cappos{cites}  Is \textit{the number of API 
calls} a good metric for security? \cappos{cite Bascule}   
\cappos{Anything else people have mentioned?}

In this paper, we provide a quantitative measure that shows
areas of privileged code are likely to have flaws.  \cappos{There was a
paper that did this using number of developers editing a module.  There
are likely others in this area that need to be part of related work.}
The measure examines lines of privileged code that are executed by popular 
programs. \yanyan{=========== attention.. please DONOT use 'touch'. It sounds 
very informal. ===========} The intuition behind this metric is that kernel code that is rarely
exercised is less likely to be well tested and is thus more likely to contain
bugs.  We examine historical privilege escalation bugs from Linux over
the past XX years and find that...

By optimizing this metric  \cappos{need a name} we devise the \emph{safely-reimplement}
architecture, which minimizes the amount of risky privileged code that is
executed.  
Risky functionality is itself implemeneted in a sandbox with
a small trusted-computing base.  
\yanyan{instead of 'risky functionatilty', should it be 'privileged operations' instead? If you 
want to quantify and minimize the risk of privilege escalation attacks (and convince me 
and others we indeed only handle privilege escalation attacks), then this makes sense.}
This additional level of sandboxing 
provides an outlet for risky functionality, without which legacy programs
will not run, while containing security flaws in this code.  
\cappos{ I need to figure out how / whether to tie this in: 
We also analyze which functionality must be retained in privileged code.}

The contributions of this paper are as follows:

Quantitative metric for security in this context\yanyan{what context?}

Novel architecture that comes from examining this metric.  Examination of
what functionality must exist where.

Implementation / Eval of sandbox using this philosophy. \yanyan{what philosophy?} 
Only X\% of zero days are exploitable w/ this vs YY\% or ZZ\%.



