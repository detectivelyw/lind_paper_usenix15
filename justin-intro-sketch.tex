\section{Introduction}
\label{sec.introduction}

[Title: Quantifying and Minimizing the Risk of Privilege Escalation Attacks.]

Despite substantial effort, modern systems are vulnerable to privilege 
escalation attacks.  [ Mention zero-days, impact, etc. ]
Decreasing the feasibility of priviledge escalation would be a substantial
step toward security as a whole.


Bugs in operating system kernels has led to the development of a diverse set of
technologies to attempt to reduce the risk.  [ OS virtualization, system
call filtering, library OS, etc. ]  Unfortunately, these technologies also 
have vulnerabilities that are also exploitable.

One contributing factor to the security problems with these new designs is
that it is not clear what portions of privileged code are
likely to be safe to expose and which would present risk.
Should one minimize the number of lines of code? [cites]  Is the number of API 
calls a good metric for security? [cite Bascule]  [Anything else people have
mentioned?]  

In this paper, we provide a quantitative measure that shows
areas of privileged code are likely to have flaws.  \cappos{There was a
paper that did this using number of developers editing a module.  There
are likely others in this area that need to be part of related work.}
The measure examines lines of privileged code that are touched by popular 
programs.  The intuition behind this metric is that kernel code that is rarely
exercised is less likely to be well tested and is thus more likely to contain
bugs.  We examine historical privilege escalation bugs from Linux over
the past XX years and find that...

By optimizing this metric [need a name] we devise the \emph{safely-reimplement}
architecture, which minimizes the amount of risky privileged code that is
executed.  
Risky functionality is itself implemeneted in a sandbox with
a small trusted-computing base.  This additional level of sandboxing 
provides an outlet for risky functionality, without which legacy programs
will not run, while containing security flaws in this code.  
\cappos{ I need to figure out how / whether to tie this in: 
We also analyze which functionality must be retained in privileged code.}

[Contributions:]

Quantitative metric for security in this context

Novel architecture that comes from examining this metric.  Examination of
what functionality must exist where.

Implementation / Eval of sandbox using this philosophy.  Only X\% of zero
days are exploitable w/ this vs YY\% or ZZ\%.



